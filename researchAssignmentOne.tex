\documentclass[12pt]{article}
\begin{document}
\title{ A REPORT ON MOBILE PHONE USE IN STAFF AND TEAM MEETINGS}
\author{WASIKE TIMOTHY,     14/U/15867/PS,     214017129}
\date{5th-04-2017}
\maketitle
\section{Executive Summary (Summary or Abstract) }
The aim of this report is to investigate staff attitudes to personal mobile phone use in staff and team meetings. A staff survey on attitudes towards the use of mobile phones in the staff / team meetings was conducted. The results indicate that the majority of staff find mobile phone use a major issue in staff meetings. The report concludes that personal mobile phones are disruptive and should be turned off in meetings. It is recommended that Organizations and institutions develop policies banning the use of mobile phones except in exceptional circumstances. 
\section{Introduction}
There has been a massive increase in the use of personal mobile phones over the past five years and there is every indication that this will continue.  Recently a number of staff have complained about the use of personal mobile phones in meetings. At present there is no official company policy regarding phone use. This report examines the issue of mobile phone usage in staff meetings and small team meetings. It does not seek to examine the use of mobile phones in the workplace at other times, although some concerns were raised. For the purposes of this report a personal mobile phone is a personally funded phone for private calls as opposed to an employer funded phone that directly relates to carrying out a particular job.
\section{Methods }
This research was conducted by questionnaire and investigated staff members’ attitudes to the use of mobile phones in staff / team meetings. A total of 53 questionnaires were distributed.. 
\subsection{Results}
The survey also allowed participants to identify any circumstances where mobile phones should be allowed in meetings and also assessed staff attitudes towards receiving personal phone calls in staff meetings in open ended questions. These results showed that staff thought that in some circumstances, e.g. medical or emergencies, receiving personal phone calls was acceptable, but generally receiving personal phone calls was not necessary. 
\section{Discussion / Interpretation of Results}
It can be seen from the results that personal mobile phone use is a problem; however t in some situations it should be permissible. 80 percent of recipients considered mobile phones to be highly disruptive and there was strong support for phones being turned off in meetings. Only 12 percent thought that mobile phone usage in staff and team meetings was not a problem, whereas 76 percent felt it was an issue. The results are consistent throughout the survey. Others felt that in exceptional circumstances mobile phones should be allowed, e.g. medical, but there should be protocols regarding this. Many companies have identified mobile phones as disruptive and have banned the use of mobile phones in meetings. Most of staff meeting time is wasted through unnecessary mobile phone interruptions. This affects time management, productivity and team focus.
\subsection{Conclusion }
The use of mobile phones in staff meetings is clearly disruptive and they should be switched off. Most staff felt it is not necessary to receive personal phone calls in staff meetings except under certain circumstances, but permission should first be sought from the team leader.
\section{Recommendations}
It is recommended that Organisations develop an official policy regarding the use of mobile phones in staff meetings. 
Mobile phones are banned in staff meetings.
Mobiles phone may be used in exceptional circumstances but only with the permission of the appropriate manager and finally, the policy needs to apply to all staff in the company.
\section{References.}
http://smallbusiness.chron.com/effects-mobile-phones-business-communication-
69543.html.

\end{document}
